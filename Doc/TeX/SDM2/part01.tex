\title{Wybrane techniki programowania gier, cz�� druga - fizyka.}
%kazdy od nowej linijki, wyrownanie do srodka
\author{\textbf{Dariusz Maciejewski}\\dmaciej1@mion.elka.pw.edu.pl\\\\Wydzia� Elektroniki i Technik Informacyjnych\\Politechnika Warszawska\\
ul. Nowowiejska 15/19\\00-665 Warszawa, Polska}
% To stworzy w�a�ciwy format tytu�u oraz spis tre�ci.
\maketitle
\date

\begin{abstract}
Opracowanie ma na celu przybli�enie zagadnienia programowania fizyki w�grach komputerowych. Dzi�ki realistycznej symulacji fizyki, gracz ma wi�kszy wp�yw na wydarzenia i�w�efekcie lepiej wczuwa si� w klimat przedstawianego �wiata. Jednak�e reprodukcja zjawisk fizycznych jest procesem bardzo kosztownym obliczeniowo, wi�c aby by�a efektywna, na�ka�dym kroku trzeba szuka� miejsc do optymalizacji kodu. W eseju om�wione s��podstawowe elementy silnika fizyki oraz rozwi�zania pozwalaj�ce na wielokrotny wzrost ich wydajno�ci.
\end{abstract}

\tableofcontents